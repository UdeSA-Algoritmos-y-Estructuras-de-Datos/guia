\documentclass[titlepage,oneside]{book}
\usepackage[utf8]{inputenc}
\usepackage{amsmath}
\usepackage{mathabx}
\usepackage{graphicx}
\usepackage{minted}
\usepackage{booktabs}
\usepackage[spanish,es-noindentfirst,es-nosectiondot,es-nolists,
es-noshorthands,es-lcroman,es-tabla]{babel}
\usepackage{lmodern}             % Use Latin Modern fonts
\usepackage[T1]{fontenc}         % Better output when a diacritic/accent is used
\usepackage[utf8]{inputenc}      % Allows to input accented characters
\usepackage{textcomp}            % Avoid conflicts with siunitx and microtype
\usepackage{microtype}           % Improves justification and typography
\usepackage[svgnames]{xcolor}    % Svgnames option loads navy (blue) colour
\usepackage[hidelinks,urlcolor=blue]{hyperref}
\hypersetup{colorlinks=true, allcolors=Navy, pdfstartview={XYZ null null 1}}
\newtheorem{lemma}{Lema}
\usepackage[width=14cm,left=3.5cm,marginparwidth=3cm,marginparsep=0.35cm,
height=21cm,top=3.7cm,headsep=1cm, headheight=1.6cm,footskip=1.2cm]{geometry}
\usepackage{csquotes}
\usepackage{fontawesome5}

\usepackage{biblatex}
\addbibresource{informe.bib}

\usepackage{multicol}

\usepackage{xcolor}

\usepackage[lastexercise,answerdelayed]{exercise}

\counterwithin{Exercise}{chapter}
\counterwithin{Answer}{chapter}
\renewcounter{Exercise}[chapter]

\def\DifficultyMarker{\faIcon{exclamation-triangle}}

\renewcommand{\ExerciseHeader}{%
\textbf{\large\ExerciseHeaderDifficulty\ExerciseName\ %
\ExerciseHeaderNB\ExerciseHeaderTitle\ExerciseHeaderOrigin}\medskip}

\newcommand{\ChatGPT}{\faIcon{robot}}

\newminted{c}{linenos,autogobble,tabsize=4}

\title{
	Algoritmos \& Estructuras de Datos\\
	\large Guía Práctica \\
	\large Universidad de San Andrés
}
\author{Magalí Marijuán\\
	\and
	I. Javier Mermet
	\and
	Matías Sandacz
}
\date{\today}

\begin{document}
\maketitle
\tableofcontents

\part{Introducción}
\chapter{Lenguaje C}

%Ejercicios sin punteros
\section{Ejercicios para entrar en calor}
\begin{Exercise}
Escribir la función que dado n $\in$ N devuelve si es primo. Recuerden que un número es primo si los únicos divisores que tiene son 1 y el mismo.
\end{Exercise}

\begin{Exercise}
Escribir la función que dado n $\in$ N devuelve la suma de todos los números impares menores que n.
\end{Exercise}

%Ejercicios para familiarizarse con punteros
\section{Punteros y Arreglos}

\begin{Exercise} ¿Cuál es el valor de \textit{a} y \textit{b} luego de ejecutar el programa?
	\newline
	\begin{ccode}
	    void myFunc(int* a, int b)
	    {
	    (*a)++;
	    b++;
	    }

	    void main()
	    {
	        int a = 10;
	        int b = 10;
	        myFunc(&a, b);
	    }
	\end{ccode}
\end{Exercise}

\begin{Exercise} ¿Que valor se imprime por consola luego de cada llamado a \texttt{printf}?
	\newline
    \begin{ccode}
        void main()
        {
           int x = 10;
           int* px = &x;

           printf("%d \n", px);
           printf("%d \n", (*px));

           (*px)++;

           printf("%d \n", px);
           printf("%d \n", (*px));
        }
    \end{ccode}
\end{Exercise}

\begin{Exercise} Implementar las siguientes funciones en C
	\Question
		\mint{c}|void crearArreglo(int v)|
		Crea un arreglo estático de enteros de tamaño 8, inicializando todos sus elementos con $v$, y lo imprime en pantalla.

	\Question \mint{c}|int* crearArregloDin(int n, int v)|
		Dado un natural $n$, crea un arreglo de enteros de ese tamaño, inicializando todos sus elementos con $v$, y devuelve un puntero al mismo.

	\Question Invocar la siguiente función con cualquiera de los arreglos inicializados anteriormente y convencerse de que sus elementos están ubicados de manera \textbf{contigua} en memoria. \textit{Recordar que cada elemento de tipo int ocupa $4$ bytes.}

    \begin{ccode}
    void mostrarMemoria(int* arr, int size)
    {
        for(int i=0; i<size; i++)
        {
            printf("Elemento: %d, Direccion: %d\n", i, &arr[i]);
        }
    }
    \end{ccode}
\end{Exercise}

\begin{Exercise}
	Cuál es la diferencia entre {\textbf{malloc}} y {\textbf{calloc}}? Cuando utilizamos la función {\textbf{free}}?
\end{Exercise}

\begin{Exercise}
	Dado el siguiente struct que representa una Persona
    \begin{ccode}
    typedef struct Persona{
        int edad;
        char* nombre;
    } Persona;
    \end{ccode}

	\mint{c}|Persona* inicializarPersonas(int n)|
	Completar la función inicializarPersonas, que dado un entero $n$, crea un arreglo de n personas, y devuelve un puntero al mismo.

	\Question Utilizando \textbf{malloc}.
 	\Question Utilizando \textbf{calloc}.
\end{Exercise}

% Ejercicios integradores con arreglos
\subsection{Más arreglos!}
\begin{Exercise} Programar las siguientes funciones en C:
	\Question \mint{c}|int maximo(int* arr, int size)|
        Dado un arreglo de enteros $arr$ de tamaño $size$, devuelve el máximo elemento.
	\Question \mint{c}|void sumador(int* arr, int size, int c)|
        Dado un arreglo de enteros $arr$ de tamaño $size$, y un entero $c$, suma $c$ a todos los elementos de $arr$.
	\Question \mint{c}|char* copiar(char* arr)|
        Crea una copia de $arr$, y devuelve un puntero a la copia.
	\Question \mint{c}|int* reverso(int* arr, int size)|
        Dado un arreglo de enteros $arr$ de tamaño $size$, devuelve su reverso. \newline
        Ejemplo: Dado [1, -2, 85, 65] se debe devolver [65, 85, -2, 1].
        \begin{enumerate}
            \item Se puede modificar $arr$.
            \item Sin modificar $arr$.
        \end{enumerate}
	\Question \mint{c}|bool estaOrdenado(int* arr, int size)|
        Dado un arreglo $arr$ de enteros de tamaño $size$, retorna true si es monótonamente creciente o monótonamente decreciente.
 	\Question \mint{c}|bool esPalindromo(char* s)|
        Dado un string $s$, retorna true si es un palíndromo. \textit{Recuerden que un palíndromo es una palabra que se lee igual en un sentido que en otro (por ejemplo; Ana, Anna, Otto).}
\end{Exercise}

% Ejercicios de searching and sorting
\section{Ordenamiento}
\begin{Exercise}
	\mint{c}|void ordenar(int* arr, int size)|
    Dado un arreglo de enteros $arr$ de tamaño $size$, escribir una función que lo ordene de menor a mayor.

	\Question Utilizando \textbf{\textit{selection sort}}.
    \Question Utilizando \textbf{\textit{insertion sort}}.
    \Question Dar la complejidad temporal en el peor caso de ambos algoritmos.
    \Question Cual es la complejidad temporal en el peor caso de ambos algoritmos, suponiendo que el arreglo de entrada ya está ordenado?
     % Para que piensen en el invariante de los algoritmos.
    % Utilizar selection sort, ya que en la iteracion k-esima, tenes los k menores elementos
    % del arreglo en las posiciones [0...k]
    \Question Suponiendo que queremos obtener únicamente los $k < size$ menores elementos del arreglo, ¿Cuál de los dos algoritmos utilizaría? ¿Por qué?
\end{Exercise}

\begin{Exercise}
Decimos que un algoritmo de ordenamiento es \textbf{estable} si preserva el orden relativo de los elementos con la misma clave. \newline

    Supongamos que queremos ordenar el arreglo de tuplas A = [(3,6), (1,5), (3,2)], utilizando como clave el primer componente de cada tupla. En este caso, dos resultados diferentes son posibles, uno de los cuales mantiene un orden relativo de elementos con claves iguales, y una en la que no:

	\definecolor{highlightgreen}{RGB}{23,81,23}
	\definecolor{highlightred}{RGB}{81,23,23}

    [(1,5), (3,6), (3,2)] \textcolor{highlightgreen}{Preserva el orden relativo entre (3,6) y (3,2)}

    [(1,5), (3,2), (3,6)] \textcolor{highlightred}{No preserva el orden relativo entre (3,6) y (3,2)}

    \Question Decidir si los algoritmos de ordenamiento presentados en el ejercicio anterior son estables. Justificar.
\end{Exercise}

\begin{Exercise}
	Consideremos el siguiente algoritmo de ordenamiento, llamado \textbf{bubble sort} (suponemos que n es el tamaño del arreglo):

    \begin{ccode}
    int i = 0;
    while( i < n-1 )
    {
        int j = 0;
        while( j < n-1 )
        {
            if( a[j] > a[j+1] )
                swap(a, j, j+1);
            j++;
        }

        i++;
    }
    \end{ccode}

    \Question Describir con palabras qué hace este algoritmo.
    \Question ¿Cuántas veces se ejecuta el swap del ciclo interior como máximo (i.e. en el peor caso)?
\end{Exercise}

\begin{Exercise}
	\mint{c}|bool twoSum(int* arr, int size, int target)|
    Dado un arreglo de enteros sin repetidos $arr$ de tamaño $size$, se debe devolver true si existen dos elementos $a1$ y $a2$ tal que $a1 + a2 = target$. En caso contrario, devolver false.

	\Question Dar una solución con complejidad temporal $O(n^2)$.
    \Question Dar una solución con complejidad temporal $O(n logn)$.

\end{Exercise}

% TODO: Deberiamos hacerlos en pseudocodigo al estilo Python? Algunos ejercicios pueden ser
% muy engorrosos de implementar. Ej: mergesort.
\chapter{Recursividad}

	\begin{Exercise}
		\mint{c}|int fibo(int n)|
	    Los números de Fibonacci son una sucesión de números naturales definida por las ecuaciones:
    	$$F_n = F_{n-1} + F_{n-2}$$ con $F_1 = F_2 = 1$. Escribir una función recursiva para computar el n-ésimo número de Fibonacci.
    \end{Exercise}

	\begin{Exercise}
		\mint{c}|int suma(int n)|
	    Escribir una función recursiva para calcular la suma de los primeros $n$ números.
	\end{Exercise}

	\begin{Exercise}
		\mint{c}|int factorial(int n)|
    	Escribir una función recursiva para calcular el factorial de $n$.
    \end{Exercise}

	\begin{Exercise}
		\mint{c}|int potencia(int a, int n)|
        Escribir una función recursiva para calcular $a^n$.
    \end{Exercise}

	\begin{Exercise}
		\mint{c}|void imprimir(int* arr, int n)|
        Escribir una función recursiva para imprimir todos los elementos de un arreglo con $n$ elementos.
    \end{Exercise}

	\begin{Exercise}
		\mint{c}|void imprimirReverso(int* arr, int n)|
        Escribir una función recursiva para imprimir todos los elementos de un arreglo con $n$ elementos en orden reverso.
    \end{Exercise}

	\begin{Exercise}
		\mint{c}|int max(int* arr, int n)|
         Escribir una función recursiva para calcular el máximo elemento de un arreglo de $n$ elementos.
    \end{Exercise}

	\begin{Exercise}
		\mint{c}|int ocurrencias(int* arr, int n, int elem)|
        Escribir una función recursiva para contar la cantidad de veces que aparece el elemento $elem$ en un arrreglo $arr$ de $n$ elementos.
	\end{Exercise}

	\begin{Exercise}
		\mint{c}|void reverso(char* str)|
        Escribir una función recursiva para convertir $str$ en su reverso. Por ejemplo, dado el String "abc", se debe modificarlo para obtener "cba".
	\end{Exercise}

	\begin{Exercise}
		Como vieron en la teórica, \textbf{merge sort} es el algoritmo recursivo más conocido para ordenar un arreglo. Utiliza una técnica algorítmica conocida como \textbf{Divide and Conquer}, y consta en dividir el arreglo en dos partes iguales, ordenar cada una recursivamente, y finalmente combinar los resultados.

		\Question Implementar merge sort utilizando pseudocódigo. Dar su complejidad temporal.
		\Question ¿Que desventaja tiene merge sort al ser un algoritmo recursivo? \newline\textit{Sugerencia: Pensar en la complejidad espacial y el stack del proceso.}
	\end{Exercise}

\begin{Exercise} La \textbf{búsqueda binaria} es un algoritmo eficiente para encontrar un elemento \textbf{en un arreglo ordenado}. El código es el siguiente: (Suponiendo que n es el tamaño del arreglo).

    \begin{ccode}
    bool binarySearch(int* a, int n, int target)
    {
        int low = 0, high = n - 1;

        // Iteramos hasta agotar los elementos
        while (low <= high)
        {
            // Calculamos el punto medio del arreglo
            int mid = (low + high)/2;

            // Encontramos el target
            if (target == nums[mid]) {
                return true;
            }

            // Si el target es menor que el elemento del medio,
            // descartamos todos los elementos a la derecha.
            else if (target < nums[mid]) {
                high = mid - 1;
            }

            // Si el target es mayor que el elemento del medio,
            // descartamos todos los elementos a la izquierda.
            else {
                low = mid + 1;
            }
        }

    // Taget no está en el arreglo.
    return false;

    }
    \end{ccode}

    \Question Describir con palabras qué hace este algoritmo.
    \Question Mostrar con un ejemplo que el algoritmo \textbf{no} es correcto si el arreglo de entrada no esta ordenado.
    \Question Implementar la búsqueda binaria utilizando recursión. Dar su complejidad temporal.
\end{Exercise}

\begin{Exercise}
    \mint{c}|int* buscarRango(int* arr, int size, int target)|
    Dado un arreglo de enteros $arr$ de tamaño $size$, ordenado de menor a mayor, encontrar los índices de comienzo y de final de $target$. Devolverlos en un arreglo. Además, se puede suponer que $target$ es un elemento de $arr$.
    \newline
    Por ejemplo, si arr = [5,7,7,8,8,8,10], target = 8, se debe devolver [3,5].

    \Question El algoritmo dado debe tener complejidad $O(n)$.
	\Question[difficulty=1] El algoritmo dado debe tener complejidad $O(\log n)$.
\end{Exercise}


\chapter{Lectura de Archivos}

% Algunos ejercicios con punteros dobles
\chapter{Magia con Punteros}

\begin{Exercise}

\begin{ccode}
typedef struct Persona {
    char* nombre;
    char* apellido;
    char* domicilio;
    int edad;
} Persona;

char* masGrande(Persona** personas, int n)
{
    // COMPLETAR
}
\end{ccode}

Completar la función $masGrande$, que dado un arreglo de n personas, devuelve el nombre de la persona de mayor edad.

\end{Exercise}

\begin{Exercise}
	\mint{c}|void imprimirMatriz(int** matrix, int n, int m)|
    Dada una matriz de $n$ filas y $m$ columnas, imprimir todos sus elementos en pantalla. Se deben imprimir todos los elementos de la primer fila, luego todos los de la segunda fila, y así sucesivamente. Por ejemplo, dada la matriz de $2x3$ [[1,2,3], [4,5,6]], se debe imprimir 1, 2, 3, 4, 5, 6.
\end{Exercise}

\begin{Exercise}
	\mint{c}|void traspuesta(int** matrix, int n)|
    Dada una matriz cuadrada de $nxn$, modificarla para obtener su traspuesta. Por ejemplo, dada la matriz cuadrada [[1,2], [3,4]], se debe modificar para obtener [[1, 3], [2, 4]]. \textit{Recuerden que la matriz traspuesta es aquella que surge como resultado de realizar un cambio de columnas por filas y filas por columnas en la matriz original.}

\end{Exercise}

\part{Estructuras de Datos}
\chapter{Listas}
\begin{Exercise}
	\Question ¿Qué cambios se le tendrían que hacer a una lista simplemente enlazada para permitir que la operación \texttt{eliminar último} tenga complejidad $O\left(1\right)$?
	\Question[difficulty=1] ¿Se le ocurre cómo hacer para que la solución al punto anterior tenga una menor complejidad de memoria?
\end{Exercise}
\begin{Answer}
		Este ejercicio apunta a ayudar a descubrir las listas doblemente enlazadas y las XOR lists.
\end{Answer}

\begin{Exercise}[origin=\ChatGPT]
	Implementar una función que permita verificar si una lista es un palíndromo (es decir, los elementos son los mismos cuando se lee de principio a fin o al revés). No se pueden usar estructuras de datos adicionales para resolver este problema.
	\Question ¿Qué complejidad temporal tiene la solución al problema anterior?
\end{Exercise}

\begin{Exercise}
	Implementar funciones para unir dos listas $l1$ y $l2$.
	\Question Anexando todos los elementos de $l2$ al final de $l1$
	\Question Intercalando elementos de $l1$ y $l2$
\end{Exercise}

\begin{Exercise}
	En diversos sistemas distribuidos es común ver arquitecturas que utilizan \textit{anillos} para balancear carga o sincronizarse. Un ejemplo es como Cassandra maneja las particiones de datos. En este ejercicio se propone:
	\Question Pensar que modificaciones hay que hacer a una lista simplemente enlazada para convertirla en una lista circular.
	\Question Queremos que nuestro anillo tenga 256 posiciones. Pensar como limitar la cantidad de nodos acorde a esta limitación.
	\Question Implementar una función \texttt{insertarDato} que permita insertar datos en nuestro anillo.

	\begin{ccode}
	int insertarDato(lista_circular_t* listac, int dato) {
		// garantizado que devuelve un valor entre 0 y 255 inclusive
		int particion = hash(dato);
		// completar
	}
	\end{ccode}

	En esta función, la \textit{partición} donde se guardará el dato es un número entero entre 0 y 255 (de modo de quedar dentro del rango de 256 posiciones de nuestro anillo). Luego, donde hay que completar, se debe buscar cuál es el nodo que maneja esa partición. Un nodo $j$ conoce su posición inicial $i_j$ y tiene una lista enlazada de datos. Una vez encontrado el nodo, se debe colocar en su lista de datos el dato pasado.
	\Question Pensar como se deben rebalancear los datos del anillo cuando se agrega un nuevo nodo en una posición aleatoria.
	\Question Análogamente, pensar como se deben rebalancear los datos del anillo cuando se remueve un nodo.
\end{Exercise}

\chapter{Pilas}
\begin{Exercise}
	Implementar un programa que lea una secuencia de letras hasta llegar a un punto e imprima los caracteres ingresados en orden inverso.
\end{Exercise}

\begin{Exercise}
	Escribir un programa que lee una secuencia de caracteres y verifica que todos los parentesis, corchetes y llaves están balanceados.
\end{Exercise}

\begin{Exercise}
	Implementar las siguientes funciones
	\Question \mintinline{c}|int stack_sorted_descending(pila_t* pila);| que verifica si los elementos de la pila están ordenados en orden decreciente
	\Question \mintinline{c}|int stack_sorted_ascending(pila_t* pila);| que verifica si los elementos de la pila están ordenados en orden creciente
\end{Exercise}

\begin{Exercise}
	Dados dos numeros $n$ y $m$, y la siguiente lista de operaciones permitidas:
	\begin{enumerate}
		\item $n \to n + 1$
		\item $n \to n + 2$
		\item $n \to n * 2$
	\end{enumerate}

	Escribir una función que encuentre la secuencia más corta de operaciones permitidas para llegar de $n$ a $m$.
\end{Exercise}

\begin{Exercise}
	Implementar una calculadora de notación polaca. A continuación, se provee un ejemplo:
\begin{ccode}

int calcular(pila_t* pila) {
	// completar
}

int main(void) {
	pila_t* pila = pila_crear();

	// 25 * 2 + 10 - 2 * 3 - 4
	// Pasa a ser:
	// - - + * 25 2 10 * 2 3 4

	pila_apilar(&pila, "4");
	pila_apilar(&pila, "3");
	pila_apilar(&pila, "2");
	pila_apilar(&pila, "*");
	pila_apilar(&pila, "10");
	pila_apilar(&pila, "2");
	pila_apilar(&pila, "25");
	pila_apilar(&pila, "*");
	pila_apilar(&pila, "+");
	pila_apilar(&pila, "-");
	pila_apilar(&pila, "-");

	int resultado = calcular(&pila);
	// debería imprimir 50
	printf("El resultado es %d", resultado);

	return 0;
}
\end{ccode}
\end{Exercise}

\chapter{Colas}
\begin{Exercise}
	Utilizando una cola, invertir una pila.
\end{Exercise}

\begin{Exercise}
Implementar una cola utilizando un arreglo
\end{Exercise}

\begin{Exercise}
Implementar una pila utilizando dos colas
\end{Exercise}

\chapter{Heaps}
\begin{Exercise}
	Implementar el algoritmo \texttt{heapify} y dar su complejidad temporal.
	\Question Utilizando \texttt{heapify up}
	\Question Utilizando \texttt{heapify down}

 
\end{Exercise}

\chapter{Árboles}

\subsection{Árboles Binarios}

\begin{Exercise}
	Recorrer un árbol binario utilizando los siguientes recorridos
	\Question Preorder
	\Question Postorder
        \Question Inorder
\end{Exercise}

\begin{Exercise}
	Dado un árbol binario, se define una \texttt{hoja} como un nodo sin hijos. Implementar una función que imprima por pantalla el valor de todas sus hojas de izquierda a derecha.
\end{Exercise}

\begin{Exercise}
    La altura de un árbol binario se define como la cantidad de ejes entre la raíz y la hoja más lejana. Implementar la función \mintinline{c}|int altura(Node* raiz)| que dada la raíz de un árbol binario, devuelve su altura. Dar su complejidad temporal.
\end{Exercise}

\begin{Exercise}
   Implemente la función \mintinline{c}|bool sumTree(Node* raíz)| que retorna verdadero si un árbol binario es un sumTree. Un sumTree es un árbol en donde el valor de todos sus nodos es igual a la suma de todos los nodos de su subárbol izquierdo más los del subárbol derecho. Dar su complejidad temporal. (Agregar un ejemplo)
\end{Exercise}

\subsection{Árboles Binarios de Búsqueda}

\begin{Exercise}
   Implemente la función \mintinline{c}|bool esBST(Node* raíz)| que retorna verdadero si el árbol binario es un árbol de búsqueda. Dar su complejidad temporal.
\end{Exercise}

\begin{Exercise}
   Implemente la función \mintinline{c}|bool pertenece(Node* raíz, target int)| que retorna verdadero si el target pertenece al árbol binario de búsqueda.
   	\Question Dar su complejidad temporal.
	\Question Dar su complejidad temporal en el caso en que el árbol esté balanceado.
\end{Exercise}

\begin{Exercise}
   Dado un árbol de búsqueda binario y un elemento, agregarlo al árbol. Se debe asegurar que el árbol resultante siga siendo un árbol de búsqueda binario válido después de agregar el elemento.
\end{Exercise}

\begin{Exercise}
   Dado un árbol de búsqueda binario y un elemento perteneciente al árbol, eliminar ese elemento del árbol. Se debe asegurar que el árbol resultante siga siendo un árbol de búsqueda binario válido después de eliminar el elemento.
\end{Exercise}

\begin{Exercise}
   Dado un arreglo entero nums donde los elementos están ordenados en orden ascendente, implementar una función que lo convierta en un árbol de búsqueda binario \textbf{balanceado}. Dar su complejidad temporal.
\end{Exercise}



\chapter{Grafos}

\chapter{Hashing}

\chapter{Respuestas}
\shipoutAnswer

\printbibliography{}

\end{document}
